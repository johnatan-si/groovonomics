The recent popularization of dynamically typed languages, such as Ruby and JavaScript, has brought more attention to the discussion about the impact of typing strategies on software development.
Types allow the compiler to find type errors sooner and potentially improve the readability and maintainability of code.
On the other hand, "untyped" code may be easier to change and require less work from programmers.
This paper tries to identify the programmers' point of view about these tradeoffs.
An analysis of the source code of 6638 projects written in Groovy, a programming language which features optional typing, shows in which scenarios programmers prefer to type or not to type their declarations. 
Our results show that types are popular in the definition of module interfaces, but are less used in scripts, test classes and frequently changed code.
There is no correlation between the size and age of projects and how their constructs are typed.
Finally, we also found evidence that the background of programmers influences how they use types.

\keywords{Type Systems, Repository Analysis, Optional Typing, Groovy}
