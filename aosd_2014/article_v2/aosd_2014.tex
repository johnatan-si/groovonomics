% LARGE SCALE

\documentclass[preprint]{sigplanconf}

\usepackage{amsmath}
\usepackage{listings}
\usepackage{float}
\usepackage{graphicx}
\usepackage{adjustbox}
\usepackage{multirow}
\usepackage{color}

\newcommand{\ts}{\textsuperscript}

\newcommand{\cell}[2][c]{\begin{tabular}[#1]{@{}c@{}}#2\end{tabular}}
\renewcommand{\arraystretch}{1.2}

\floatstyle{ruled}
\newfloat{Listing}{tbp}{loa}

\begin{document}

\special{papersize=8.5in,11in}
\setlength{\pdfpageheight}{\paperheight}
\setlength{\pdfpagewidth}{\paperwidth}

\conferenceinfo{CONF 'yy}{Month d--d, 20yy, City, ST, Country} 
\copyrightyear{20yy} 
\copyrightdata{978-1-nnnn-nnnn-n/yy/mm} 
\doi{nnnnnnn.nnnnnnn}

% Uncomment one of the following two, if you are not going for the 
% traditional copyright transfer agreement.

%\exclusivelicense                % ACM gets exclusive license to publish, 
                                  % you retain copyright

%\permissiontopublish             % ACM gets nonexclusive license to publish
                                  % (paid open-access papers, 
                                  % short abstracts)

%\titlebanner{banner above paper title}        % These are ignored unless
%\preprintfooter{short description of paper}   % 'preprint' option specified.

\title{How Do Programmers Use Optional Typing? An Empirical Study}

\authorinfo{Carlos Souza}
           {Software Engineering Lab\\Federal University of Minas Gerais (UFMG)}
           {carlosgsouza@gmail.com}
\authorinfo{Eduardo Figueiredo}
           {Software Engineering Lab\\Federal University of Minas Gerais (UFMG)}
           {figureido@dcc.ufmg.br}

\maketitle

\begin{abstract}
The recent popularization of dynamically typed languages, such as Ruby and JavaScript, has brought more attention to the dynamic vs static type systems discussion.
The usage of types allows the compiler to find type errors sooner and potentially improve the readability and maintainability of a program.
On the other hand, dynamically typed languages may require less work from programmers and allow them to change their code more easily.
This paper tries to identify the programmers' points of views about these tradeoffs.
An analysis of the source code of 6638 projects written in Groovy, a programming language which features optional typing, shows in which situations programmers prefer typing or not their declarations. 
Results suggest that the need for maintainability, frequency of change and programmers background are important factors in this decision.
\end{abstract}

\category{D.2.3}{Software Engineering}{Cooding Tools and Techniques}

\terms
Experimentation, Language

\keywords
Type systems, static analysis, Groovy

\section{Introduction}
When choosing a programming language for a project, a developer considers several characteristics of that language.
One of the most important of these characteristics is its type system, which can be static or dynamic.
The type system determines when the type of a statement is defined \cite{types_and_programming_languages}. 
Statically typed languages, such as Java and C\#, require programmers to define the type of a declaration, which can then be used by the compiler to check for type errors. 
On the other hand, in dynamically typed languages, such as Ruby and JavaScript, the definition of the type of a statement only happens at run time.

Discussions about what is the best type system for a particular situation have become increasingly important in recent years due to the rapid popularization of dynamically typed languages. 
According to the TIOBE Programming Community Index \cite{tiobe}, a well-known ranking that measures the popularity of programming languages, 27\% of the programming languages used in industry are dynamically typed. 
A decade ago, this number was only 17\%. 
Among the 10 languages on top of the ranking, four are dynamically typed: JavaScript, Perl, Python and PHP. 
None of these languages were among the top 10 rank until 1998.

Several factors may be considered when choosing between a dynamically or statically typed language. 
Since the type system of dynamically typed languages is simpler, they tend to allow programmers to code faster \cite{types_and_programming_languages} and adapt to frequently changing requirements more easily \cite{gradual_typing}.
Also, by removing the repetitive work of defining types, these languages allow programmers to focus on the problem to be solved rather than on the rules of the language \cite{dynamically_typed_languages}.

Statically typed languages also have their advantages. 
For instance, they allow compilers to find type errors statically \cite{should_your_specification_language_be_typed}. 
Typed declarations increase the maintainability of systems because they implicitly document the code, telling programmers about the nature of statements \cite{type_systems,mayer2012static}. 
Systems built with these languages tend to be more efficient since they do not need to perform type checking during execution \cite{bruce2002foundations,jit}. 
Finally, modern development environments, such as Eclipse and IDEA, are able to assist programmers with functionalities such as code completion based on the information provided by statically typed declarations \cite{bruch2009learning}.

Some languages try to combine characteristics from both static and dynamic type systems.
Groovy \cite{groovy} is one of these languages.
Although Groovy is mostly a dynamically typed language, it gives programmers the option to use type annotations as a means to document their code.
It is also possible to turn static type checking on so the compiler can find type errors before execution.
This allows developers to choose the most appropriate paradigm for each situation.

Understanding the point of view of programmers about the tradeoffs between type paradigms is an important matter.
Programmers can make an informed decision by knowing which languages provide the right benefits for their particular context.
Programming language developers can consider this information in their design so they can develop the most appropriate features for their target audience.
Finally, tools can be developed or improved to overcome any weaknesses of a given language. 

This paper presents a large scale empirical study about how programmers use optional typing in Groovy in order to understand which factors actually influence the decision of a developer for using types or not. 
This question was studied based on the analysis of a massive dataset with almost seven thousand Groovy projects.
Through a static analysis of these projects, it was possible to understand when developers use types and then extract what are the factors that influence this decision.

Results show that programmers consider types as a means to document their code.
This is even more evident on the definition of the interface of modules.
Conversely, when neither readability nor stability is a concern, programmers tend to type their declarations less often.
In addition to that, programmers seem to prefer the flexibility of untyped declarations in frequently changed code.
Finally, the experience of a programmer with other languages has a relevant influence on his or her choice for typing a declaration or not.

The remainder of this paper is organized as follows. 
Section \ref{groovy} introduces the main concepts of the Groovy programming language and Section \ref{settings} presents the study settings.
Section \ref{results} describes the results of the study, which are then discussed in Section \ref{discussion}.
Threats to the validity and related work are presented in Sections \ref{threats} and \ref{related}.
Finally, Section \ref{conclusion} concludes this study and suggests future work.


% TODO include these references in the intrudoction
% [4] BRUCE, K. B. Foundations of object-oriented languages:
% types and semantics. MIT Press, Cambridge, MA, USA, 2002


% [4] BRUCE, K. B. Foundations of object-oriented languages:
% types and semantics. MIT Press, Cambridge, MA, USA, 2002







%
%  THE GROOVY LANGUAGE
%

\section{The Groovy Language\label{groovy}}
Groovy is a dynamically typed programming language designed to run on the Java Virtual Machine.
Its adoption has grown remarkably over the last years.
According to the TIOBE Programming Index, Groovy is the 22\textsuperscript{nd} most popular language in the software industry \cite{tiobe}, ahead of languages like Prolog, Haskell and Scala. 
It builds upon the strengths of Java, but has additional features inspired by dynamic languages such as Python and Ruby.
Like Java, Groovy code is compiled to bytecode, allowing it to seamlessly integrate with existing Java classes and libraries. 
These factors have attracted a large number of Java programmers who want to use Groovy's dynamic functionality without having to learn a completely different language or change the execution platform of their systems. 

% Groovy was designed to be more expressive and concise than Java.
% Two implementations of a simple algorithm are shown below.
% Given a list of numbers, return a list containing only the even numbers of that list.
% Listing \ref{javaClass} shows the Java implementation while Listing \ref{groovyClass} shows the Groovy counterpart. 
% Because of its high level of expressiveness, Groovy is able to reduce much of the boilerplate required in Java.
% Listing \ref{groovyClass} shows that Groovy offers a native syntax for lists (lines 3, 6 and 14) and operator overloading (line 6). 
% Semicolons are optional, except when there are multiple statements in the same line. 
% When the keyword $return$ is omitted (line 10), the last expression evaluated with a method is returned. 
% Also, parentheses in method calls can often be omitted (line 16).
% In addition, Groovy implicitly imports frequently used classes, like those of the $java.util$ package, and methods, like $System.out.println$ (line % 16).

% \begin{Listing}[ht]
% \begin{lstlisting}[language=Java,tabsize=2,breaklines=true,numbers=left]
% import java.util.ArrayList;
% import java.util.List;
% 
% public class JavaFilter {
% 	List<Integer> evenNumbers(List<Integer> list) {
% 		List<Integer> result = new ArrayList<Integer>();
% 		for(int item : list) {
% 			if(item % 2 == 0) {
% 				result.add(item);
% 			}
% 		}
% 
% 		return result;
% 	}
% 
% 	public static void main(String[] args) {
% 		List<Integer> list = new ArrayList<Integer>();
% 		list.add(1);
% 		list.add(2);
% 		list.add(3);
% 		list.add(4);
% 
% 		List<Integer> result = new JavaFilter().evenNumbers(list);
% 		System.out.println(result);
% 	}
% }
% \end{lstlisting}
% \caption{A simple algorithm written in Java}
% \label{javaClass}
% \end{Listing}
% 
% \begin{Listing}[ht]
% \begin{lstlisting}[language=Java,tabsize=2,breaklines=true,numbers=left]
% class GroovyFilter {
% 	List<Integer> evenNumbers(List<Integer> list) {
% 		List<Integer> result = []
% 		for(int item : list) {
% 			if(item % 2 == 0) {
% 				result << item
% 			}
% 		}
% 
% 		result
% 	}
% 
% 	public static void main(String[] args) {
% 		List<Integer> list = [1, 2, 3, 4]
% 		List<Integer> result = new GroovyFilter().evenNumbers(list)
% 		println result
% 	}
% }
% \end{lstlisting}
% \caption{A simple algorithm written in Groovy}
% \label{groovyClass}
% \end{Listing}

% The design of Groovy was influenced by dynamic features of programming languages such as Ruby and Python.
% Listing \ref{dynamicInfuence} shows how these features can be used to rewrite the same algorithm presented % in Listing \ref{javaClass} in a single line of code. 
% First, notice that the code shown in Listing \ref{dynamicInfuence}  is a script, rather than a class file.
% It makes use of a closure to allow a programmer to define a filter logic.
% This closure is passed down to the \emph{findAll} method, which  apply this closure to every element of % the list to in order to decide if that element should be returned or not.
% Closures are one of the most important features of Groovy as compared to Java.
% They allow a functional programming style, which is both expressive and powerful.
% 
% \begin{Listing}[ht]
% \begin{lstlisting}[language=Java,tabsize=2,breaklines=true,numbers=left]
% println([1, 2, 3, 4].findAll {it % 2 == 0})
% \end{lstlisting}
% \caption{A class written in Groovy}
% \label{dynamicInfuence}
% \end{Listing}
% 
% Metaprogramming is another dynamic feature present in Groovy. 
% Listing \ref{metaprogramming} shows how to add a method to an existing class dynamically.
% By adding the method \emph{evenNumbers()} to the \emph{List} class, it is possible to achieve higher % expressiveness.
% This is specially useful when implementing Domain Specific Languages \cite{fowler10}.
% 
% \begin{Listing}[ht]
% \begin{lstlisting}[language=Java,tabsize=2,breaklines=true,numbers=left]
% List.metaClass.evenNumbers = {
% 	delegate.findAll {it % 2 == 0}
% }
% println([1, 2, 3, 4].evenNumbers())
% 
% \end{lstlisting}
% \caption{An example of metaprogramming in Groovy}
% \label{metaprogramming}
% \end{Listing}% 

When Groovy was first launched, in 2007, it was a purely dynamically typed language.
However, it allowed programmers to optionally type their declarations.
Examples of typed and untyped declarations combined in the same file are shown in Listing \ref{dynamicTyping}.
This kind of typing, however, should not be confused with static typing since the Groovy compiler does not use these type annotations to look for errors.
For example, the snippet of code shown in Listing \ref{typeError} compiles without any errors.
During runtime, the \emph{string} variable references an instance of the \emph{Integer} class.
However, an exception is thrown when we try to invoke the method \emph{toUpperCase} since the \emph{Integer} class does not have this method.

\begin{Listing}[ht]
\begin{lstlisting}[language=Java,tabsize=2,breaklines=true,numbers=left]
class DynamicTyping {
	private String typedField
	private untypedField

	DynamicTyping(typedParam) {}

	def untypedMethod(untypedParam, int typedParam) {
		def untypedVariable = 1.0
		return untypedVariable
	}

	int typedMethod()  {
		String typedVariable = ""
		return typedVariable
	}
}
\end{lstlisting}
\caption{Groovy is a dynamic language}
\label{dynamicTyping}
\end{Listing}

\begin{Listing}[ht]
\begin{lstlisting}[language=Java,tabsize=2,breaklines=true,numbers=left]
String string = new Integer(1)
string.toUpperCase()
\end{lstlisting}
\caption{A class written in Groovy}
\label{typeError}
\end{Listing}

Since version 2.0, Groovy allows programmers to explicitly activate static typing with the usage of the \emph{@TypeChecked} annotation.
This makes Groovy a gradually typed language \cite{gray05,gray08,gray11,siek07,takikawa12}.
In this mode, the Groovy compiler looks for type errors and fails if it finds any.
Listing \ref{staticTyping} shows an example of static typing in Groovy.
Trying to compile the class \emph{TypeCheckedGroovyClass} produces an error since the method \emph{sum} is supposed to receive two parameters of the type \emph{int}, but it is actually called with two parameters of the type \emph{String}.

\begin{Listing}[ht]
\begin{lstlisting}[language=Java,tabsize=2,breaklines=true,numbers=left]
@TypeChecked
class TypeCheckedGroovyClass {
	
	static int sum(int a, int b) {
		a + b
	}

	public static void main(String[] args) {
		println sum("1", "2")
	}
}
\end{lstlisting}
\caption{A class written in Groovy}
\label{staticTyping}
\end{Listing}

The \emph{@TypeChecked} annotation is reasonably recent and most Groovy programmers still do not use it. 
Typing annotations on the other hand are very popular.
Although they do not provide static type checking, they are capable of documenting the code and aiding in the integration with development tools.
In the remainder of this text, we refer to declarations with type annotations as "typed", while the word "untyped" is used for declarations with no type annotations.







%
% STUDY SETTINGS
%

\section{Study Settings\label{settings}}
The study presented in this paper consists in the static analysis of the source code of a corpus of 6638 Groovy projects.
Its goal is to find which the factors have an actual influence over the decision of a developer for typing or not a declaration. 
In this section we present the research questions we want to answer, the data collection and analysis procedures as well as the characterization of the studied dataset.

\subsection{Research Questoins\label{questions}}
We propose the following research questions about the points of views of programmers about the usage of types.

\begin{itemize}
	\item \textbf{Question Q1: Do programmers use types more often in the interface of their modules?} We believe that the benefits of the use of types are more clear in declarations that define the interface of modules. In such cases programmers are specifying how the rest of the program should interact with a module, potentially causing positive impacts on the readability and maintainability. On the other hand, in declarations that are hidden from external modules programmer could benefit from the simplicity and flexibility offered by dynamic typing. 
	
	\item \textbf{Question Q2: Do programmers use types less often in test classes and scripts?} Many studies analyze typing strategies in the main classes of a program. However, little is known about how programmers use types scripts or test classes. We want to understand if programmers consider different typing strategies in these scenarios.

	\item \textbf{Question Q3: Does the previous experience of programmers with other languages influence their choice for typing their code?} We believe that programmers familiar with a dynamically typed language are more confortable with the lack of types and end up using types less often in Groovy. 

	\item \textbf{Question Q3: Does the size, age or level of activity have any influence on the usage of types?} We hypothesize that as these metrics grow, there is an increased concern about keeping code more maintainable. This can potentially lead programmers to use types more often as a means to improve code readability.

	\item \textbf{Question Q5: In frequently changed code, do developers prefer typed or untyped declarations?} It makes sense to assume that developers try to increase the maintainability of frequently changed code. One way to achieve that is improving the readability of such code with the use of types. On the other hand, the flexibility of untyped declarations is capable of increasing the changeability of those files. We want to understand which one of these strategies is actually preferred by developers in the end.

\end{itemize}

\subsection{Data Collection Procedure\label{dataCollection}}
The projects used in this study were obtained from GitHub, a popular source control service based on Git.
For each project, it was necessary to retrieve its source code, metadata, commit history, and the metadata of all of its developers.
GitHub does not offer a listing of all hosted projects, but it offers two search mechanisms, a REST API and a web based search page.
Unfortunately the GitHub API is too limited for our requirements.
It imposes a limit of one thousand results and does not allow filtering projects by their programming language.

In order to retrieve an extensive dataset, it was necessary to write a bot to simulate human interactions with the GitHub webpage and search for projects. 
Some special care was necessary to make this work. 
For instance, because the number of results is limited to 1 thousand projects, we had to segment the queries.
Multiple requests were made, and each one asking for the name of all projects created on a given month.
Results were then combined into a single list.
Another problem faced was that GitHub denies excessive requests from the same client.
By adding a 10 seconds delay between requests, it was possible to overcome this limitation.

With the name of all projects in hands, it was then possible to use the GitHub REST API to query their metadata.
That metadata also contains the identifiers of the developers and of the commits of that project.
Using those identifiers we once again used GitHub REST API and obtained the background of all developers and the file changes of all projects.


%DATASET
\subsection{Dataset\label{dataset}}

% count scripts, main classes and test classes

Our dataset consists of 6638 projects with almost 9.8 million lines of code.
Table \ref{tab:dataset_characterization} shows descriptive statistics for the size, age and number of commits of these projects.
There are more than 1.5 million declarations of all types and visibilities in our dataset.
More details about these declarations are shown in Table \ref{tab:number_of_declarations} and \ref{tab:number_of_declarations_by_visibility}.
Note that most fields are declared with private visibility while most methods are public.
This is explained by the fact that, different from Java, in Groovy, the default visibility for fields is private and public for methods.

\begin{table}[h!]

\centering{}%
\renewcommand{\arraystretch}{1.2}

\begin{tabular}{|c|c|c|c|c|c|}
\hline
{}		& Mean	& Median	& \cell{Sd}	& Max	& Total		\\
\hline
\hline
Size (LoC)	& 1471 	& 529  & 4545  & 149933	& 9770783	\\ \hline
Commits   	& 31  	& 5    & 175   & 6545		& 203375	\\ \hline
Age (Days)  & 361  	& 280  & 333   & 1717		& 2395441	\\ \hline
\end{tabular}
\caption{Dataset characterization}
\label{tab:dataset_characterization}
\end{table}

\begin{table}[h!]

\centering{}%
\renewcommand{\arraystretch}{1.2}

\begin{tabular}{|c|c|c|c|c|c|}
\hline
{}		& Mean	& Median	& Sd & Max	& Total		\\
\hline
\hline
\cell{Field}                      	&   54    &  19  &    163  &    5268 & 366148   \\ \hline                  
\cell{Constructor\\Parameter}     	&    3    &   0  &     16  &     933 &  18956   \\ \hline                  
\cell{Method\\Parameter}          	&   30    &   6  &    110  &    3554 & 202617   \\ \hline                  
\cell{Method\\Return}             	&   53    &  15  &    165  &    4893 & 357997   \\ \hline                  
\cell{Local\\Variable}            	&   88    &  21  &    361  &   16427 & 602645   \\ \hline                  
\hline																			
Public                    			&   74    &  20  &    239  &    7942 & 507296   \\ \hline                  
Protected                 			&    6    &   0  &     32  &    1394 &  42646   \\ \hline                  
Private                   			&   58    &  21  &    178  &    5268 & 395776   \\ \hline                  
\hline																			
\cell{All\\Declarations} &  227    &  71  &    744  &   29862 &1548363   \\ \hline                  
\end{tabular}
\caption{Number of Declarations}
\label{tab:number_of_declarations}
\end{table}


More than 4 thousand developers were involved in the projects in our dataset.
While 96\% of the projects were developed by small groups of 3 people or less, there were projects with up to 58 people.
These developers have different backgrounds.
Figure \ref{fig:other_languages} shows what are the most popular languages used by these developers in other GitHub projects. 
Java is the most popular among them.
About 40\% of the developers of projects in our dataset also have Java projects hosted on GitHub.

\begin{table}[h!]

\centering{}%
\renewcommand{\arraystretch}{1.2}

\begin{tabular}{|c|c|c|c|}
\hline
{}									& Private	& Protected	& Public	\\
\hline
\hline
\cell{Field}                      	&   346462 &    2996 &   16690   		\\ \hline                  
\cell{Constructor Parameter}     	&   680 &     246 &   18030   		\\ \hline                  
\cell{Method Parameter}          	&   27174 &  10897 & 164546 		\\ \hline                  
\cell{Method Return}             	&   21460 	&  28507 	& 308030 \\ \hline                  
\hline																			
\end{tabular}
\caption{Number of Declarations by Visibility}
\label{tab:number_of_declarations_by_visibility}
\end{table}

% Figure \ref{fig:typeSystem_background} shows what is the type system of the languages the developers in our dataset have experience with.
% Most of them have experience with both statically and dynamically typed languages.
% There are two small groups however that have experience with only one type system outside Groovy.
 
% \begin{table}[ht]
% \caption{Distribution of the Number of Developers in a Project}
% \centering{}%
% \begin{tabular}{|c|c|c|}
% \hline 
% Number of Developers & Fraction of Projects\tabularnewline
% \hline 
% \hline 
% 1 & 84\%\tabularnewline
% \hline 
% 2 & 9\%\tabularnewline
% \hline 
% 3 & 3\%\tabularnewline
% \hline 
% 4 or more & 4\%\tabularnewline
% \hline 
% \end{tabular}
% \label{tab:number_of_developers}
% \end{table}

\begin{figure}[h]
\centering \includegraphics[width=0.45\textwidth]{../analysis/result/languages.png}
\caption{Most popular languages among Groovy developers}
\label{fig:other_languages} 
\end{figure}

% \begin{figure}[ht]
% \centering \includegraphics[width=0.45\textwidth]{typeSystem_background}
% \caption{Type System of other languages used by programmers}
% \label{fig:typeSystem_background} 
% \end{figure}



%ANALYSIS
\subsection{Analysis\label{analyzer}}
In order to understand where programmers use types, we developed a static code analyzer based on the Groovy metaprogramming library.
It is capable of retrieving the declaration information of parameters and returns of methods, parameters of constructors, fields and local variables.
In addition, the analyzer can tell if a declaration is part of a test class or a script and what is its visibility.

A relevant decision we made was not to compile projects, which would require all dependencies to be resolved.
This is not feasible given the size of our dataset.
Instead, we generated the AST for each file using the $CONVERSION$ phase of the Groovy compiler.
At this phase, the compiler has not tried to resolve any dependencies yet, but it is capable of generating an AST with enough information to determine if a a declaration is typed or not.
This makes it possible to analyze each Groovy file separately without having to compile the whole project.

The downside of the approach described above is that we can not analyze Groovy code in conjunction with its dependencies. 
For example, it is impossible to determine whether programmers tend to type code that interacts with other typed modules since we have not resolved any dependencies to these modules.
However, our choice was fundamental in order to execute a study with such an extensive dataset.
Nevertheless, as shown in the next section, we were still able to obtain detailed and relevant results.











%
% RESULTS
%
\section{Results\label{results}}
This sections presents the results obtained form the analysis of 6638 Groovy projects.
We show how the usage of types varies according to declaration types and visibilities, nature of code, programmers' background, project maturity and frequency of changes.
In Section \ref{discussion}, these results will be discussed in order to try and answer the research questions proposed in this paper.

\subsection{Overall Result\label{sec:results-overall}}
Figure \ref{fig:all_histogram_all} shows a histogram and the descriptive statistics for the relative usage of types in declarations of projects.
This value can vary from 0 (a project does not declare any types) to 1 (all declarations of a project are typed). 
All declarations are considered.
Note that there is a significant number of projects for which the relative usage of types is either approximately 0 or 1.
These are mostly small projects.
About 95\% of them have less than 1000 lines of code and 22\% of them have less than 100 lines of code.
In such projects, it is easier to be consistent on the typing strategy since there are fewer declarations.
We initially considered not to include these projects in the rest of our analysis since they could not be represent well the entire population of Groovy projects.
However, doing so did not alter the results significantly and we decided to include all projects in our analysis regardless of their size.
In the rest of this section, this data will be presented in more details so we can understand which factors lead programmers to use types or not.

\begin{figure}[h]
\centering 
\includegraphics[width=0.45\textwidth]{../analysis/result/all/histograms/5_all_types.png} 

\vspace{0.3cm}
\small
\begin{tabular}{|c|c|c|cccc|}
\hline
{}		&  {}		&  {}			&  \multicolumn{4}{c|}{Quartiles}				\\
n		& mean	& std. dev.	& 1\ts{st}	& 2\ts{nd}	& 3\ts{rd}	& 4\ts{th}		\\
\hline
\hline
6638 	& 0.45	& 0.28		& 0.25	& 0.42		& 0.64	& 1.00		\\
\hline
\end{tabular}


\caption{Usage of types in all declarations of all projects}
\label{fig:all_histogram_all} 
\end{figure}



% TYPE
\subsection{Declaration Type\label{sec:results-type}}
This section investigates whether programmers use types differently depending on the type of the declaration.
For each project, we measured the relative usage of types by declaration type, which can be a field, a constructor parameter, the return or a parameter of a method or a local variable.
These results are displayed in box plots in Figure \ref{fig:all_boxplot_type} along with the corresponding descriptive statistics.
Note that the size of each sample, \emph{n}, is different since not all projects have all types of declarations. 
For instance, note that there are only 1670 out of 6638 projects with declarations of constructor parameters.
Meanwhile, 6000 projects have declarations of fields.

\begin{figure}[h]
\centering 
\includegraphics[width=0.45\textwidth]{../analysis/result/all/boxplots/6_declarations_by_type.png} 
\vspace{0.3cm}

\renewcommand{\arraystretch}{1.2}
\small
\begin{tabular}{|c|c|c|c|c|}
\hline
Declaration Type		& n		& mean	& median		& std. dev.	\\
\hline
\hline
Field					& 6000	& 0.43	& 0.39		& 0.33		\\ \hline
Constructor Parameter	& 1670	& 0.80	& 1.00		& 0.35		\\ \hline
Method Parameter		& 4867	& 0.67	& 0.86		& 0.36		\\ \hline
Method Return			& 5881	& 0.68	& 0.75		& 0.31		\\ \hline
Local Variable			& 5845	& 0.29	& 0.18		& 0.32		\\ \hline
\end{tabular}

\caption{Usage of types in all declarations by type of declaration}
\label{fig:all_boxplot_type} 
\end{figure}


% TODO
% Explain why types are less typed
% privateField                   20 5985    0.42    0.33    0.38    0.40    0.35   0      1      1  0.39    -0.94  0.00
% protectedField                 21  413    0.78    0.36    1.00    0.85    0.00   0      1      1 -1.40     0.31  0.02
% publicField                    22  953    0.89    0.29    1.00    0.98    0.00   0      1      1 -2.52     4.69  0.01

The results presented in Figure \ref{fig:all_boxplot_type} suggest that programmers use types differently depending on the type of a declaration.
Local variables, for example, are typed less often.
Half of the projects have only 18\% or less of their local variables typed.
Conversely, methods and constructors are typed in most of the cases.
Note that the median for constructor parameters is equal to 1.00, which means that, at least, half of the projects with constructor parameters type all declarations of this kind.

The box plot graph and the descriptive statistics are not enough to determine whether the difference in the usage of types in any two types of declaration is significant.
In order to do that, a significance test should be applied. 
We start by defining a hypothesis below, which can then be rejected or accepted.

\begin{description}
\item[H0] The data for two declaration types are drawn from the same population, i.e., there is no difference in how programmers type different types of declarations
\item[H1] Programmers type their declarations differently depending on the type of the declaration
\end{description}

The appropriate significance test should be chosen carefully.
We can not assume that the data presented in Figure \ref{fig:all_boxplot_type} follows any particular distribution.
In fact, an analysis of the box plots for declarations of local variables, methods and constructors shows that these samples are heavily skewed and thus do not follow a Normal distribution.
Because of this, parametric tests such as the \emph{t-test} can not be applied.

A valid alternative for our scenario is the paired Mann-Whitney U test, which works on ranks and does not require an underlying Normal distribution.
It computes a \emph{p-value} indicating whether two samples are significantly different from each other.
The smaller the value of \emph{p}, the "more significant" is the difference and, consequently, the stronger the rejection of the null hypothesis.
Typically values of 0.05 and 0.01 are used to determine if the null hypothesis can be rejected or not.
In this study, we use 0.001 for this purpose.
This value might seem too small at first, which would require the difference between two samples to be too high in order to consider them different from each other.
However, due to the large size of our dataset, this \emph{p-value} seems reasonable \cite{labovitz68}.
In addition to the \emph{p} value, the Mann-Whitney U test also reports a confidence interval describing the difference of the medians of the two  populations.
This interval can be used to measure how different are two samples.
All of the confidence intervals in this study are calculated with a confidence of 99\%.

% Use a super small significance level
% Labovitz, Sanford. "Criteria for selecting a significance level: A note on the sacredness of. 05." The American Sociologist 3.3 (1968): 220-222.

\begin{table}[ht]

\centering{}%
\renewcommand{\arraystretch}{1.2}

\begin{tabular}{|c|c|c|c|}
\hline 
								& {}					& p  	      & conf. interval\\
\hline
\hline
\multirow{4}{*}{Field}			& Constructor Parameter	& 0.0000 & (-0.50, -0.44)\\
								& Method Parameter		& 0.0000 & (-0.32, -0.27)\\
								& Method Return			& 0.0000 & (-0.30, -0.26)\\
								& Local Variable 		& 0.0000 & ( 0.13, 0.17)\\
\hline
{}								& Field					& 0.0000 & (0.44, 0.50)	\\
Constructor						& Method Parameter		& 0.0000 & (0.06, 0.09)	\\
Parameter						& Method Return			& 0.0000 & (0.08, 0.13)	\\
{}								& Local Variable		& 0.0000 & (0.62, 0.67)	\\
\hline
{}								& Field					& 0.0000 & ( 0.27,  0.32)	\\
Method							& Constructor Parameter	& 0.0000 & (-0.09, -0.06)	\\
Parameter						& Method Return			& 0.0054 & ( 0.00,  0.00)	\\
{}								& Local Variable		& 0.0000 & ( 0.44,  0.50)	\\
\hline
{}								& Field					& 0.0000 & ( 0.26,  0.30)	\\
Method							& Constructor Parameter	& 0.0000 & (-0.13, -0.08)	\\
Return							& Method Parameter		& 0.0054 & ( 0.00,  0.00)	\\
{}								& Local Variable		& 0.0000 & ( 0.44,  0.47)	\\
\hline
{}								& Field					& 0.0000 & (-0.17, -0.44) \\
Local							& Constructor Parameter	& 0.0000 & (-0.67, -0.27) \\
Variable						& Method Parame			& 0.0000 & (-0.50, -0.26) \\
{}								& Method Return			& 0.0000 & (-0.47,  0.17) \\
\hline  
\end{tabular}
\caption{Mann-Whitney U Test results for the comparison between the usage of types by type of declaration}
\label{tab:all_utest_type}
\end{table}

The significance test results are displayed in Table \ref{tab:all_utest_type}.
It shows the \emph{p} value and confidence interval reported by the Mann-Whitney U test for every two declaration types.
%The confidence interval shows what is the difference of means between the declaration types shown in the first and second columns.
%For example, the confidence interval of the first row indicates with a confidence level of 99\% that the difference between the medians of fields %and constructor parameters is between $-0.50$ and $-0.44$.
There are only two types of declaration for which \emph{p} values different from zero are reported, parameters and returns of methods.
This allows us to reject the null hypothesis and consider the usage of types in these two declarations not significantly different.
This is reasonable since returns and parameters of methods are declared together as part of a method signature.
All other declaration types, however, can be considered significantly different from each other.
This includes parameters of methods and constructors.
Although these are essentially the same type of declaration in Groovy, they seem to be perceived differently by programmers when it comes to typing.
Nevertheless, the difference between these two declaration types reported by the confidence interval is relatively small, $(-0.09, -0.06)$.
Other interesting insights provided by these results are that local variables are the least typed declarations while constructor parameters are the most typed ones.

% VISIBILITY
\subsection{Declaration Visibility\label{sec:results-visibility}}

% TODO In fact, almost all projects type all of their protected declarations.
This sections presents an analysis about how programmers type their declarations according to the their visibilities.
We follow the same approach from the previous section.
Figure \ref{fig:all_boxplot_visibility_all} shows box plots for the usage of types per declaration type along with the descriptive statistics.
Declarations of fields, parameters and returns of methods and parameters of constructors of all projects are considered.
Mann-Whitney U test results are presented in Table \ref{tab:all_utest_visibility}.

\begin{figure}[h]
\centering 
\includegraphics[width=0.45\textwidth]{../analysis/result/all/boxplots/23_declarations_by_visibility.png} 


\vspace{0.3cm}

\renewcommand{\arraystretch}{1.2}
\small
\begin{tabular}{|c|c|c|c|c|}
\hline
Declaration Visibility	& n		& mean	& median	& std. dev.	\\
\hline
\hline
Public    				& 5852	& 0.69	& 0.75		& 0.29		\\ \hline
Protected 				& 2387	& 0.93	& 1.00		& 0.19		\\ \hline
Private   				& 6023	& 0.43	& 0.40		& 0.32		\\ \hline
\end{tabular}
\caption{Usage of types in all declarations by type of declaration}
\label{fig:all_boxplot_visibility_all} 
\end{figure}

The null hypothesis can be rejected for all cases since the reported \emph{p} values are always equal to $0$.
Protected declarations are those typed most often.
Notice how skewed is the distribution for these elements in Figure \ref{fig:all_boxplot_visibility_all}.
Almost all 2387 projects which use protected visibility in their declarations have all of their protected fields, methods and constructors typed.
The confidence intervals reported by the U tests show a large difference between these declarations and those with either private or public visibility.
Although public declarations are not typed as much, they are also typed very often.
At least half of the projects type 75\% or more of their declarations.
Conversely private declarations are those with the smallest relative use of types.

\begin{table}[h]

\centering{}%
\renewcommand{\arraystretch}{1.2}

\begin{tabular}{|c|c|c|c|}
\hline 
								& {}		& p		& conf. interval	\\
\hline
\hline
\multirow{2}{*}{Public}			& Protected	& 0.0000 & (-0.22, -0.18)	\\ 
								& Private	& 0.0000 &  (0.28, 0.31)	\\
\hline  
\multirow{2}{*}{Protected}		& Public	& 0.0000 & (0.18, 0.22)	\\
								& Private	& 0.0000 & (0.54, 0.57)	\\
\hline  
\multirow{2}{*}{Private}		& Public	& 0.0000 & (-0.31, -0.28)	\\
								& Protected	& 0.0000 & (-0.57, -0.54)	\\
\hline  
\end{tabular}
\caption{Mann-Whitney U Test results for the comparison between the usage of types by visibility of declaration}
\label{tab:all_utest_visibility}
\end{table}


% BREAKING VISIBILITY BY DECLARATION TYPE
% Figures \ref{fig:all_boxplot_visibility_methodReturn}-\ref{fig:all_boxplot_visibility_field} detail these results by declaration type.
% \begin{figure}[h]
% \centering 
% \includegraphics[width=0.45\textwidth]{../analysis/result/all/boxplots/11_returns_of_methods.png} 
% \caption{Usage of types in declarations of returns of methods by visibility}
% \label{fig:all_boxplot_visibility_methodReturn} 
% \end{figure}
% 
% \begin{figure}[h]
% \centering 
% \includegraphics[width=0.45\textwidth]{../analysis/result/all/boxplots/14_parameters_of_methods.png} 
% \caption{Usage of types in declarations of parameters of methods by visibility}
% \label{fig:all_boxplot_visibility_methodParameter} 
% \end{figure}
% 
% \begin{figure}[h]
% \centering 
% \includegraphics[width=0.45\textwidth]{../analysis/result/all/boxplots/17_parameters_of_constructors.png} 
% \caption{Usage of types in declarations of constructor parameters by visibility}
% \label{fig:all_boxplot_visibility_constructorParameter} 
% \end{figure}
% 
% \begin{figure}[h]
% \centering 
% \includegraphics[width=0.45\textwidth]{../analysis/result/all/boxplots/20_fields.png} 
% \caption{Usage of types in declarations of fields by visibility}
% \label{fig:all_boxplot_visibility_field} 
% \end{figure}





% TESTS
\subsection{Test Classes and Main Classes\label{sec:results-tests}}
We now analyze the use of types in test classes in comparison to main classes.
A simple heuristic to determine if a class was written as a test class or a main class is used.
In Groovy, like in Java, it's common to organize test classes and main classes in different source folders.
The convention adopted by popular build tools among Groovy programmers, such as Gradle and Maven, assumes test classes and main classes are in the \emph{src/test/groovy} and \emph{src/main/groovy} directories respectively.
Projects written in Grails, an extremely popular web framework in the Groovy community, use a slightly different convention and organize the test classes in the \emph{test} directory.
Based on these conventions, we can assume that any classes inside a \emph{test} directory, but not in a \emph{main} directory, are test classes.

\begin{figure}[h]
\centering 
\includegraphics[width=0.45\textwidth]{../analysis/result/test/comparison/boxplots/6_declarations_by_type.png} 
\vspace{0.1cm}
\renewcommand{\arraystretch}{1.2}
\small
\raggedleft
\begin{tabular}{|c|c|c|c|c|c|}
\hline
\cell{Declaration\\Type}	& \cell{Class\\Type}& n		& mean		& median	& \cell{std.\\dev.}\\
\hline
\hline
\multirow{2}{*}{Field}	& Test			& 1769	& 0.48 		& 0.47	& 0.43 \\ 
                		& Main			& 5857	& 0.43 		& 0.39	& 0.33 \\
\hline                
Constructor 			& Test			&  124	& 0.77 		& 1.00	& 0.41 \\                
Parameter				& Main			& 1623	& 0.80 		& 1.00	& 0.34 \\                
\hline
Method      			& Test			& 1524	& 0.34 		& 0.00	& 0.43 \\                
Parameter				& Main			& 4593	& 0.71 		& 0.91	& 0.35 \\                
\hline
Method      			& Test			& 4334	& 0.85 		& 1.00	& 0.31 \\                
Return		         	& Main			& 5299	& 0.54 		& 0.60	& 0.39 \\                
\hline
Local					& Test			& 2842	& 0.23 		& 0.00	& 0.35 \\                
Variable	        	& Main			& 5548	& 0.30 		& 0.19	& 0.32 \\                
\hline
\end{tabular}

\caption{Usage of types by declaration type in test classes and main classes}
\label{fig:test_boxplot_type} 
\end{figure}

For every project, we measured the usage of types in test classes and main classes.
Script files are not considered in this analysis.
We found test classes in 4350 of the 6638 projects in our dataset.
Results are displayed in Figure \ref{fig:test_boxplot_type} and show the relative usage of types by declaration type.
White and gray box plots correspond to test classes and main classes respectively.
We also applied the Mann Whitney U test, but this time however we are not comparing declaration types to each other.
Instead, we are comparing declarations in test classes to declarations in main classes.
Results for the U test are shown in Table \ref{tab:test_utest_type}.
We do not show results grouped by visibility since this is usually not a concern when writing test classes.

According to Figure \ref{fig:test_boxplot_type} and Table \ref{tab:test_utest_type}, the usage of types in test classes is very different for declarations of local variables and methods.
While local variables in main classes are not typed very often, they are typed even less in test classes.
At least half of the projects type none of the declarations of this kind in test classes.
The difference in declarations of parameters of methods is even more evident since they are often typed in main classes, but almost never typed in test classes.
The confidence interval reported by the U test in this case is $(-0.533, -0.420)$.
The large width of the box plots for fields and method parameters is noteworthy.
This indicates that many projects type either almost all or none of these declarations.

Curiously, method returns are significantly more typed in test classes.
The difference reported by the confidence interval in Table \ref{tab:test_utest_type} for this case is (0.200, 0.250).
At least half of the projects type all of their method returns in test classes.
Although counter intuitive this result can be easily explained.
Automated testing frameworks usually enforce a certain method signature for test methods.
JUnit for example, which is used in 2525 of the 4350 projects with test classes, requires test methods to be typed as \emph{void}.
Other popular test frameworks, such as TestNG, have similar requirements.
This implies that, in this case, developers type their methods not because they want to, but because they need to.


\begin{table}[h!]
\centering{}%
\begin{tabular}{|c|c|c|}
\hline 
Declaration Type 		& p & conf. interval \\
\hline 
\hline 
Field                             & 0.0020		& (0.000, 0.000) \\ \hline
Constructor Parameter  & 0.4389	& (0.000, 0.000) \\ \hline
Method Parameter         & 0.0000	& (-0.533, -0.420) \\ \hline
Method Return              & 0.0000	& (0.200, 0.250) \\ \hline
Local Variable               & 0.0000		& (-0.070, -0.032) \\ 
\hline 
\end{tabular}
\caption{Mann-Whitney U Test results for the comparison between the usage of types by test classes and main classes}
\label{tab:test_utest_type}
\end{table}

% SCRIPTS
\subsection{Script Files and Class Files\label{sec:results-scripts}}
This section investigates how programmers type their code in script files.
Similar to what was done in the previous section, we measured the usage of types in script files and class files in all projects and compared the obtained data.
Determining whether a file corresponds to a script or a class is fairly simple.
In Groovy, a script file is compiled into a class extending \emph{groovy.lang.Script}.
We consider all classes extending this class to be scripts.

Figure \ref{fig:script_boxplot_type} and Table \ref{tab:script_utest_all} show the results of our analysis.
Note that constructors and fields are not considered since there is no way to declare those elements in scripts.
Also, we do not present an analysis of declarations grouped by visibility since, although allowed, defining the visibility of a declaration inside a script does not make any sense.

\begin{figure}[h]
\centering 
\includegraphics[width=0.45\textwidth]{../analysis/result/script/comparison/boxplots/6_declarations_by_type.png} 
\vspace{0.1cm}
\renewcommand{\arraystretch}{1.2}
\small
\raggedleft
\begin{tabular}{|c|c|c|c|c|c|}
\hline
\cell{Declaration\\Type}&\cell{File\\Type}	& n		& mean		& median	& \cell{std.\\dev.}\\
\hline
\hline	
Method					& Script			& 504 	& 0.40	& 0.23	& 0.42	 \\                
Parameter					& Class				& 4647	& 0.69	& 0.86	& 0.35	 \\                \hline	
Method					& Script			& 583 	& 0.34	& 0.00	& 0.43	 \\                
Return					& Class				& 5662	& 0.70	& 0.77	& 0.30	 \\                \hline	
Local					& Script			& 1775	& 0.28	& 0.07	& 0.37	 \\                
Variable	 				& Class				& 5246	& 0.30	& 0.18	& 0.32	 \\                
\hline
\end{tabular}
\caption{Usage of types by declaration type in script files and class files}
\label{fig:script_boxplot_type} 
\end{figure}

Programmers type all kinds of declarations differently in scripts.
According to the \emph{p} values shown in Table \ref{tab:script_utest_all}, the null hypothesis can be rejected for all cases.
The confidence interval reported for local variable declarations though is relatively small.
This indicates a small difference for this type of declaration between scritps and classes.
On the other hand, declarations of methods present a very different behavior.
Unlike in classes, most of these declarations are not typed in scripts.
This is even more clear in method returns, where most of projects do not type any of these declarations in script files.
Note however that the value for the last quartile of these declarations is very high, superior to 0.8.
This indicates that, although most projects prefer not to use types in these declarations, there are a few projects that consistently type most of them.

\begin{table}[ht]
\centering{}%
\begin{tabular}{|c|c|c|}
\hline 
Declaration Type & p & conf. interval \\
\hline 
\hline 
Method Parameter              & 0.0000	& (-0.385, -0.234) \\ \hline
Method Return                 & 0.0000	& (-0.500, -0.400) \\ \hline
Local Variable                & 0.0000	& (-0.021, 0.000)  \\ \hline
\end{tabular}
\caption{Mann-Whitney U Test results for the comparison between the usage of types in script files and class files}
\label{tab:script_utest_all}
\end{table}



% PROGRAMMERS BACKGROUND
\subsection{Programmers' Background\label{sec:results-background}}
In this section, we analyze how programmers use types in their declarations according to their backgrounds.
Projects are distributed in three groups based on the type system of the languages their developers have used on GitHub.
The first group comprises those projects of programmers who developed only in statically typed languages, such as Java or C\#.
The projects of those who developed only in dynamically typed languages, such as Ruby or JavaScript, comprise the second group.
Finally, the third group is formed by the projects of those programmers with both dynamically and statically typed languages in their portfolio.
Figures \ref{fig:background_boxplot_type} and \ref{fig:background_boxplot_visibility}  show results by declaration type and visibility respectively.
Results for the Mann-Whitney U test are reported in Tables \ref{tab:background_utest_type} and \ref{tab:backgorund_utest_visibility}.
Notice that these tables are divided in three groups, each one corresponding to the comparison between the data of two of the three different groups.

\begin{figure}[h]
\centering 
\includegraphics[width=0.45\textwidth]{../analysis/result/background/comparison/boxplots/6_declarations_by_type.png} 
\vspace{0.1cm}
\renewcommand{\arraystretch}{1.2}
\small
\raggedleft
\begin{tabular}{|c|c|c|c|c|c|}
\hline
\cell{Declaration\\Type}						& \cell{Background}	& N		& Mean		& Median	& \cell{Std.\\Dev.}\\
\hline
\hline
{}												& Static			& 782	& 0.56		& 0.52 		& 0.35\\			 
Field											& Both				& 3183	& 0.43		& 0.39 		& 0.34\\			 
{}												& Dynamic			& 2035	& 0.38		& 0.36 		& 0.29\\
\hline								 			
\multirow{3}{*}{\cell{Constructor\\Parameter}}	& Static			& 224	& 0.83		& 1.00 		& 0.33\\			 
												& Both				& 991	& 0.80		& 1.00 		& 0.35\\			 
												& Dynamic			& 455	& 0.80		& 1.00 		& 0.34\\			 
\hline								
\multirow{3}{*}{\cell{Method\\Parameter}}		& Static			& 662	& 0.73		& 0.91 		& 0.34\\			 
												& Both				& 2694	& 0.67		& 0.84 		& 0.36\\			 
												& Dynamic	 		& 1511	& 0.65		& 0.83 		& 0.37\\	
\hline						 			
\multirow{3}{*}{\cell{Method\\Return}}			& Static			& 764	& 0.73		& 0.85 		& 0.30\\			 
												& Both				& 3205	& 0.66		& 0.75 		& 0.32\\			 
												& Dynamic	 		& 1912	& 0.68		& 0.74 		& 0.29\\ 
\hline											
\multirow{3}{*}{\cell{Local Variable}}			& Static			& 798	& 0.39		& 0.31 		& 0.36\\			 
												& Both				& 3230	& 0.28		& 0.17 		& 0.32\\			 
												& Dynamic			& 1817	& 0.25		& 0.14 		& 0.30\\ 
\hline
\end{tabular}
\caption{Usage of types by declaration type and programmer background}
\label{fig:background_boxplot_type}
\end{figure}

                
\begin{figure}[ht]
\centering 
\includegraphics[width=0.45\textwidth]{../analysis/result/background/comparison/boxplots/23_declarations_by_visibility.png} 
\vspace{0.1cm}
\renewcommand{\arraystretch}{1.2}
\small
\raggedleft
\begin{tabular}{|c|c|c|c|c|c|}
\hline
\cell{Declaration\\Type}						& \cell{Background}	& n		& mean		& median	& \cell{std.\\dev.}\\
\hline
\hline
{}												& Static			& 757 	& 0.73		& 0.83 		& 0.29\\			 
Public											& Both				& 3191	& 0.68		& 0.75 		& 0.30\\			 
{}												& Dynamic			& 1904	& 0.69		& 0.71 		& 0.27\\
\hline								 								    	    		    		    						
\multirow{3}{*}{\cell{Protected}}				& Static			& 287 	& 0.92		& 1.00 		& 0.20\\			 
												& Both				& 1275	& 0.94		& 1.00 		& 0.18\\			 
												& Dynamic			& 825 	& 0.93		& 1.00 		& 0.20\\			 
\hline																    	    		    		    					
\multirow{3}{*}{\cell{Private}}					& Static			& 787 	& 0.56		& 0.53 		& 0.34\\			 
												& Both				& 3196	& 0.43		& 0.40 		& 0.33\\			 
												& Dynamic	 		& 2040	& 0.38		& 0.37 		& 0.28\\	
\hline						 			
\end{tabular}
\caption{Usage of types by declaration visibility and programmer background}
\label{fig:background_boxplot_visibility}
\end{figure}	

\begin{table}[h!]
\centering{}%
\renewcommand{\arraystretch}{1.2}

\begin{tabular}{|c|c|c|c|}
\hline 
{}															& Declaration Type 		& p 		& conf. interval \\
\hline 
\hline 
\multirow{5}{*}{\cell{Static\\vs.\\Static and\\Dynamic}}	& Field					& 0.0000	& (0.073, 0.163)	\\ \cline{2-4}
															& Constructor Parameter	& 0.7281	& (0.000, 0.000)	\\ \cline{2-4}
															& Method Parameter		& 0.0000	& (0.000, 0.036)	\\ \cline{2-4}
															& Method Return			& 0.0000	& (0.000, 0.067)	\\ \cline{2-4}
															& Local Variable		& 0.0000	& (0.038, 0.100)	\\ 
\hline 
\hline

\multirow{5}{*}{\cell{Static\\vs.\\Dynamic}}				& Field					& 0.0000	& (0.139, 0.228)	\\ \cline{2-4}
															& Constructor Parameter	& 0.9347	& (0.000, 0.000)	\\ \cline{2-4}
															& Method Parameter		& 0.0000	& (0.005, 0.076)	\\ \cline{2-4}
															& Method Return			& 0.0000	& (0.000, 0.062)	\\ \cline{2-4}
															& Local Variable		& 0.0000	& (0.057, 0.126)	\\ 
\hline 
\hline
\multirow{5}{*}{\cell{Static and\\Dynamic\\vs.\\Dynamic}}	& Field					& 0.0000	& (0.000, 0.050)	\\ \cline{2-4}
															& Constructor Parameter	& 0.7531	& (0.000, 0.000)	\\ \cline{2-4}
															& Method Parameter		& 0.0032	& (0.000, 0.013)	\\ \cline{2-4}
															& Method Return			& 0.5269	& (0.000, 0.000)	\\ \cline{2-4}
															& Local Variable		& 0.0071	& (0.000, 0.005)	\\ 
\hline 
\end{tabular}
\caption{Mann-Whitney U Test results for the comparison between the usage of types by declaration type and programmers background}
\label{tab:background_utest_type}
\end{table}
      
\begin{table}[h!]
\centering{}%
\renewcommand{\arraystretch}{1.2}
\begin{tabular}{|c|c|c|c|}
\hline 
{}															& Declaration Visibility 	& p 		& conf. interval \\
\hline 
\hline 
\multirow{3}{*}{\cell{Static vs.\\Static and\\Dynamic}}		& Public					& 0.0000	& (0.000, 0.058)	\\ \cline{2-4}
															& Protected					& 0.1287	& (0.000, 0.000)	\\ \cline{2-4}
															& Private					& 0.0000	& (0.082, 0.167)	\\ 
\hline 
\hline

\multirow{3}{*}{\cell{Static\\vs.\\Dynamic}}				& Public					& 0.0000	& (0.000, 0.065)	\\ \cline{2-4}
															& Protected					& 0.1130	& (0.000, 0.000)	\\ \cline{2-4}
															& Private					& 0.0000	& (0.143, 0.229)	\\ 
\hline 
\hline
\multirow{3}{*}{\cell{Static and\\Dynamic vs.\\Dynamic}}	& Public					& 0.0000	& (0.000, 0.051)	\\ \cline{2-4}
															& Protected					& 0.8260	& (0.000, 0.000)	\\ \cline{2-4}
															& Private					& 0.8328	& (0.000, 0.000)	\\ 
\hline 
\end{tabular}
\caption{Mann-Whitney U Test results for the comparison between the usage of types by declaration visibility and programmers background}
\label{tab:backgorund_utest_visibility}
\end{table}
              
The usage of types by programmers with both statically and dynamically typed languages in their portfolio is similar to that of programmers with dynamically languages only.
According to Tables \ref{tab:background_utest_type} and \ref{tab:backgorund_utest_visibility}, the only significant differences in the usage of types are in  declarations of fields or in those with public visibility.
Still, the difference indicated by the confidence interval in these cases is arguably small, between 0 and 0.05 in both cases.

Programmers with only statically typed languages in their portfolio tend to use more types than the others.
There are only two situations where the \emph{p} value reported by the Mann-Whitney U test is big enough so the null hypothesis can not be rejected, protected declarations and declarations of constructor parameters.
In all other cases, these programmers use type more often than the others.

% It is important to say that 75\% of the programmers with projects in this group only have Java projects in GitHub besides their Groovy projects.
% This suggests that these projects are more typed 


% PROJECT MATURITY
\subsection{Project Maturity\label{sec:results-maturity}}
This section investigates whether programmers use types differently in their code depending on the project maturity.
In this study, we consider three metrics to define the maturity of a project: age, number of lines of code and number of commits.
The latter measures the level of activity of the project.
We start by analyzing the correlation between these metrics and the relative use of types in declarations by type and visibility.
The Spearman rank correlation coefficient is used for this purpose.
This coefficient, which can go from -1 to 1, is a measure of the dependence between two variables.
A positive value means that two variables are correlated, i.e, as the value of one grows, so does the value of the other.
A negative value means an inverse correlation.
Values close to 1 or -1 indicate very strong relationships while values above 0.5 or below -0.5 can be considered strong correlations.

\begin{table}[h!]
\centering{}%
\begin{tabular}{|c|c|c|c|c|}
\hline 
Declaration Type/Visibility	& Size		& Age	&	Commits\\
\hline 
\hline 
Field						&  0.221	& -0.063	&  0.153	\\ \hline
Constructor Parameter		& -0.072	& -0.132	& -0.053	\\ \hline
Method Parameter			& -0.123	& -0.079	& -0.004	\\ \hline
Method Return				& -0.071	&  0.168	& -0.027	\\ \hline
Local Variable		 		&  0.057	& -0.049	&  0.112	\\ 
\hline 			 
\hline 		 
Public						& -0.063	&   0.119	& -0.024	\\ \hline
Protected					& -0.286	&  -0.020	& -0.165	\\ \hline
Private				 		&  0.213	&  -0.068	&  0.160	\\ \hline
\end{tabular}
\caption{Spearman Correlation between the usage of types and the size, age and number of commits of projects}
\label{tab:all_correlation_maturity}
\end{table}	
% TODO characterization

Table \ref{tab:all_correlation_maturity} shows the Spearman correlation coefficient between the usage of types and the three metrics of maturity.
Most of values in this table are close to 0.
There are only a few which could indicate a very weak relationship, such as \emph{Size} vs. \emph{protected} or \emph{commits} vs. \emph{private}.
All in all, we can not say that there is a correlation between the metrics of maturity and the usage of types in Groovy projects.
\begin{table}[h!]

\centering{}%
\renewcommand{\arraystretch}{1.2}

\begin{tabular}{|c|c|c|c|c|c|}
\hline
{}		& mean	& median	& std. dev.	& max	& total		\\
\hline
\hline
Size (LoC)	& 9947 	& 5627 & 14594  & 149933	& 2218189	\\ \hline
Commits   	& 487  	& 213    & 800   & 6545		& 108583	\\ \hline
Age (Days)  & 600  	& 574  & 350   & 1469		& 133697	\\ \hline
\end{tabular}
\caption{Descriptive statistics for mature projects}
\label{tab:mature_dataset_characterization}
\end{table}

The lack of correlation does not necessarily imply that the maturity of a project has no influence on the usage of types.
A possibility is that this influence appears only in the most mature projects, where the values of all of these three metrics are high enough.
In order to determine whether this is true, we conduct now a comparison between the most mature projects and the rest of the dataset.
We define a mature project as a project that is 100 days old or more and has, at least, 2KLoC and 100 commits.
These numbers were defined by manually inspecting our dataset and finding that there are popular and mature projects that barely exceed these three metrics.
According to our criteria, there are 223 mature projects in our dataset, which are detailed in Table \ref{tab:mature_dataset_characterization}.


Figures \ref{fig:size_boxplot_type}  and \ref{fig:size_boxplot_visibility} show the box plots for the usage of types in mature projects and others by declaration type and visibility respectively.
Mann-Whitney U test results for the samples presented in these graphs are displayed in Table \ref{tab:size_utest_type+visibility}.
There are significant differences in the usage of types only in two cases, declarations of fields and protected declarations.
For private declarations, although the value of \emph{p} is $0$, the confidence interval for the difference of medians is negligible.
These differences are too small, specially when compared to the differences found when analyzing programmers' background, script files and test classes.
Hence, we can not conclude that programmers type their declarations differently depending on the maturity of a project, at least, when considering our definition of maturity, based on size, age and number of commits.

\begin{figure}[h!]
\centering 
\includegraphics[width=0.45\textwidth]{../analysis/result/size/comparison/boxplots/6_declarations_by_type.png} 
\vspace{0.1cm}
\renewcommand{\arraystretch}{1.2}
\small
\raggedleft
\begin{tabular}{|c|c|c|c|c|c|}
\hline
\cell{Declaration\\Type}	& \cell{Project\\Type}& n	& mean	& median	& \cell{std.\\dev.}\\
\hline
\hline
\multirow{2}{*}{Field}		& Mature		& 221 	& 0.53 		& 0.48	& 0.27 \\
							& Other			& 5779	& 0.43 		& 0.39	& 0.33 \\                
\hline                																								
Constructor					& Mature		& 172 	& 0.83 		& 1.00	& 0.30 \\                
Parameter	 				& Other			& 1498	& 0.80 		& 1.00	& 0.35 \\                
\hline																								
Method						& Mature		& 222 	& 0.69 		& 0.78	& 0.29 \\                
Parameter      				& Other			& 4645	& 0.67 		& 0.86	& 0.37 \\                
\hline																								
Method		         		& Mature		& 222 	& 0.72 		& 0.79	& 0.24 \\                
Return						& Other			& 5659	& 0.68 		& 0.75	& 0.32 \\                
\hline																								
Local						& Mature		& 223	& 0.32 		& 0.22	& 0.28 \\                
Variable					& Other			& 5622 	& 0.29 		& 0.17	& 0.32 \\                
\hline																								
\end{tabular}
\caption{Usage of types in projects by declaration type and project maturity}
\label{fig:size_boxplot_type} 
\end{figure}


\begin{figure}[h!]
\centering 
\includegraphics[width=0.45\textwidth]{../analysis/result/size/comparison/boxplots/23_declarations_by_visibility.png} 
\vspace{0.1cm}
\renewcommand{\arraystretch}{1.2}
\small
\raggedleft
\begin{tabular}{|c|c|c|c|c|c|}
\hline
\cell{Declaration\\Type}			& \cell{Project\\Type}	& n		& mean		& median	& \cell{std.\\dev.}\\
\hline
\hline
\multirow{2}{*}{\cell{Public}}		& Mature				& 223 	& 0.72		& 0.76 		& 0.24\\
{}									& Other					& 5629	& 0.69		& 0.75 		& 0.29\\			 
\hline								 							  	  			  			  							
\multirow{2}{*}{\cell{Protected}}	& Mature				& 183 	& 0.88		& 1.00 		& 0.21\\			 
									& Other					& 2204	& 0.94		& 1.00 		& 0.19\\			 
\hline															  	  			  			  						
\multirow{2}{*}{\cell{Private}}		& Mature				& 221 	& 0.53		& 0.48 		& 0.26\\			 
									& Other					& 5802	& 0.43		& 0.40 		& 0.32\\			 
\hline						 			
\end{tabular}
\caption{Usage of types in projects by declaration visibility and project maturity}
\label{fig:size_boxplot_visibility}
\end{figure}	

\begin{table}[h!]
\centering{}%
\begin{tabular}{|c|c|c|}
\hline 
Declaration Type/Visibility 		& p & conf. interval \\
\hline 
\hline 
Field                  & 0.0000	& (0.053,0.173)	\\ \hline
Constructor Parameter  & 0.3203	& (0.000,0.000) \\ \hline
Method Parameter       & 0.0794	& (-0.051,0.002) \\ \hline
Method Return          & 0.9373	& (-0.031,0.051) \\ \hline
Local Variable         & 0.0002	& (0.012,0.078)	\\ \hline
\hline 
Public		& 0.8540	& (-0.031,0.045)	\\ \hline
Private		& 0.0000	& (0.000,0.000)		\\ \hline
Protected	& 0.0000	& (0.055,0.169)		\\ 
\hline 
\end{tabular}
\caption{Mann-Whitney U Test results for the comparison between the usage of types by mature projects and others}
\label{tab:size_utest_type+visibility}
\end{table}	





% FREQUENCY OF CHANGES
\subsection{Frequency of changes\label{sec:results-changes}}
This section investigates whether programmers prefer to type their declarations in frequently changed code or not.
Only the mature projects defined in the previous section are considered since we would not be able to obtain meaningful results from small and young projects.

We found that in most of the mature projects programmers prefer untyped declarations in frequently changed code.
We calculated the Spearman correlation coefficient between the frequency of changes of a file and the usage of types in that file for all mature projects.
In projects where programmers prefer the usage of types in frequently changed files, this coefficient is positive, and negative in case programmers prefer untyped declarations.
Figure \ref{fig:change_spearman} displays the cumulative distribution of this coefficient across the mature projects dataset.
It shows that 65\% of them present a negative Spearman correlation coefficient and that almost half of these present strong correlations, i.e., inferior to -0.5.
On the other hand, it can be said that only 10\% of the mature projects present strong positive correlations.

\begin{figure}[h]
\centering \includegraphics[width=0.45\textwidth]{../analysis/result/change_commits_distribution.png} 
\caption{Spearman ranking for the correlation between frequency of changes of files and the usage of types in mature projects}
\label{fig:change_spearman} 
\end{figure}






















%
% DISCUSSION
%
\section{Discussion\label{discussion}}
In this section, we discuss the results of our study in order to try and answer the research questions proposed in Section \ref{questions}.
Although we were able to get a good understanding of the usage of types in different contexts, the cause for such results is still unclear in many of them.
We provide several hypothesis with the goal of identifying future research topics that can provide more detailed insights about such causes.

% Q1
\subsection*{Q1: Do programmers use types more often in the interface of their modules?\label{discussion-q1}}
The analysis of the usage of types by kind and visibility of declarations, presented in Sections \ref{sec:results-type} and \ref{sec:results-visibility}, show evidence that the answer for Q1 is affirmative.
Methods and constructors are the most typed declarations.
In particular, constructors, which are extremely important members of a module definition, have their parameters always typed in most projects.
Fields can also be part of a module definition, but are significantly less typed than methods and constructors.
This can be explained by the fact that, in Groovy, similar to what happens with Java, interactions with fields of other modules usually happen through accessor methods.
This is supported by Section \ref{dataset}, which shows that the great majority of field declarations are private.

Public and protected declarations, which compose the interface of a module, are typed more often than private declarations
It is interesting to notice that protected declarations are typed with a frequency much higher than public declarations.
These declarations give subclasses and other classes in the same package access to internal elements of a class, forming a tightly coupled relationship \cite{Chidamber94}.
In addition, protected methods are frequently designed to be overwritten by subclasses.

While it is clear that module definitions are typed more often, the cause for such phenomena is still open to discussion.
We believe that the main motivation for this is the implicit documentation provided by types.
In these scenarios, types provide useful hints about the behavior of modules \cite{Curtis1987} and define pre and post condition of contracts
\cite{Meyer88, Meijer04, Wadler04, Plosch97, Flanagan2006, Furr09}.
Users of an well defined module learn how to use it faster and do not need to read its implementation to understand how to use it.
Delicate contracts, such as those defined by protected methods, may require more documentation and thus are typed more often.

We can also speculate that declarations that are not part of a module definition, local variables or those with private visibility, require less documentation and that is why they are typed less often.
Programmers can easily find all the references to these elements.
Local variables are only used inside the block of code where they were declared, while all the references to elements declared privately are in the same file.
This makes it easier for them to infer the types of such declarations even when it is not explicitly defined.

Documentation is not be the only reason why programmers type declarations in modules interfaces.
We can think of at least other two.
First, a programmer might type a declaration so he or she can get code assistance from the development environment.
For example, typing the declaration of a method parameter allows the development environment to provide code completion for that parameter inside the method.
Another possibility is that, even though Groovy is actually a dynamically typed language, programmers might type their declarations thinking that the compiler will check for type errors, which would lead to safer interactions between modules


% Q2
\subsection*{Q2: Do programmers use types less often in test classes and scripts?\label{discussion-q2}}
There are notable differences between the usage of types in either test classes or scripts and the main classes of a program.
Sections \ref{sec:results-tests} and \ref{sec:results-scripts} show that, in these scenarios, programmers use types less often in declarations of methods and local variables.
The only exceptions are declarations of method returns in test classes, which are typed in most cases due to requirements of testing frameworks such as JUnit and TestNG.

If we are right about our hypothesis that programmers type their modules as a means to document their code, than this could explain this less frequent use of types.
Scripts and test classes are usually not designed as reusable modules.
Test classes have the sole goal of verifying a program's functionality and not interfering with it while scripts can not be instantiated or referenced by other modules.
In these scenarios, programmers might perceive documentation as less important.
It is curious however that test code itself is usually considered by programmers as a form of documentation \cite{Beck03,Meyerovich13}.
Because of this, we were expecting programmers to actually use more types in test classes as a means to improve the documentation provided by them.
Perhaps, although programmers use test classes as documentation, they might not write them with this goal in mind.

An alternative explanation for the fact that scripts and tests are typed less frequently is that most of them are probably simpler than functional classes.
As found in the recent work of Hanenberg et al, typed code potentially has positive impacts on the development time of easier tasks \cite{Hanenberg13}.
In such case, programmers might not type their declarations in scripts or test classes since this would allow them to finish their tasks faster.

% Q3
\subsection*{Q3: Does the previous experience of programmers with other languages influence their choice for typing their code?\label{discussion-q3}}
The analysis presented in Section \ref{sec:results-background} indicates that the answer for Question Q4 is affirmative.
The choice for using types on a language with optional typing, such a Groovy, is in fact influenced by the programmers' experience with other languages.
However, this influence is only visible on programmers who have only static languages on their GitHub portfolio.
In most cases, these programmers type their declarations more often than others.

The usage of types by those who developed only in dynamically typed languages and those with experience in both typing paradigms is not significantly different.
Apparently programmers who develop in a language with a dynamic type system get used to the lack of types, leading them to declare types less often.
This hypothesis supports the work of Daly et al which suggests that such programmers have ways of reasoning about types that compensate for the lack of static type information \cite{ruby_vs_druby}.

% Q4
\subsection*{Q4: Does the size, age or level of activity have any influence on the usage of types?\label{discussion-q4}}
We initially believed that as these metrics grow, the maintenance of projects would become more difficult, leading programmers to use more types as a means to make code more readable.
However, the analysis presented in Section \ref{sec:results-maturity} shows no evidence of such behavior.
We consider two hypothesis in order to explain these results.
First, the considered metrics might not actually correlate to the desire of programmers to improve the readability of their code.
Second, even if these metrics are a good indicative of the necessity of better maintainability, once projects start growing and aging, programmers might not have the opportunity or desire to make their code more maintainable.

% Q5
\subsection*{Q5: In frequently changed code, do developers prefer typed or untyped declarations?\label{discussion-q5}}
In frequently changed code, there are arguments in favor and against using types.
Since types act as documentation, programmers might use them to make code more maintainable and easier to change  \cite{should_your_specification_language_be_typed}.
On the other hand, untyped code is simpler and can potentially be changed faster \cite{gradual_typing}.
The results presented in Section \ref{sec:results-changes} however show that the latter is considered more often than the first.

In most projects, the usage of untyped declarations grows as the frequency of changes in a file increases.
Apparently, programmers understand that untyped code makes maintenance tasks easier.
One can argue that the causal relationship is the opposite, i.e., these files have to be changed more often because of problems generated by the use of untyped declarations.
However, it is very unlikely that programmers would not notice that untyped declarations would be causing such problems and would not add types to those declarations in order to fix them.










%
% THREATS TO VALIDITY
%
\section{Threats to Validity\label{threats}}
In this section, we discuss potential threats to validity. As usual, we have arranged possible threats in two categories, internal and external validity \cite{Wohlin2012}. 

\subsection*{Internal Validity}
Perhaps the most relevant internal threat to our study is that in a large scale empirical study such as ours, there might be many confounds which are difficult to identify.
In Section \ref{sec:results-background} we consider that the GitHub portfolio of a programmer represents well his experience with other languages and type systems, but this might not be true for all programmers.
They may have projects in their portfolio that they have not worked on or projects hosted somewhere else written in other languages.
There is still the possibility of a programmer having multiple GitHub accounts with different languages in each one, causing this programmer to be measured twice with different inferred backgrounds.
Due to the large number of programmers considered in our study, we expect these special cases not to have a large influence over the results.

There are other factors which might have influenced programmers besides the ones considered in this study.
Some frameworks require programmers to use typed or untyped declarations in some cases.
For example, we found that the data collected in test classes is biased by the fact that popular testing frameworks, such as JUnit, require test methods to have their returns declared as \emph{void}.
There might be other similar cases that we were not able to indentify.
In addition to that, the programmers' background, which we have shown to have influence over programmers, might interfere with the analysis of other influencing factors.

In our discussion, in Section \ref{discussion}, we avoided providing definitive answers for what caused the behavior we found in our analysis.
Instead, we raised many hypothesis to explain these results.
It is difficult however to state whether these hypothesis make sense or not.
In order to overcome this uncertainty, we are planning a series of controlled and qualitative studies which will complement the findings in this study in order to validate or reject these hypothesis.

\subsection*{External Validity}
Although we have analyzed a very extensive number of Groovy projects, it can not be said that we have all possible scenarios covered.
By manually inspecting our dataset, we could find only a few projects with characteristics of software developed inside an organization.
Most of them were developed by small groups of people or open source communities.
It is probable that such enterprise projects are hosted privately on GitHub or in private servers, hence unavailable to us.

The behavior observed for Groovy projects can be very different in other languages.
Most languages are not like Groovy and feature either static or dynamic typing, forcing programmers to choose a single typing strategy for all scenarios in a single project.
Even a language with a hybrid typing paradigm might implement different strategies which will be perceived differently by the programmers of that language.
Finally, the tools used to code in a given language might influence programmers to chose different type strategies.











%
% RELATED WORK
%
\section{Related Work\label{related}}
There are multiple studies in the literature which compare different typing strategies.
Although we are not aware of any studies that analyze this question using a large scale case study as ours, we know of multiple controlled experiments with significant results.
One of the goals of this paper is to complement theses results by providing a a different point of view to this question. 

In a recent experiment, Hanenberg et al. studied the impact of the use of types on the development time of programmers while performing tasks on an undocumented API \cite{Hanenberg13}.
The experiment divided 27 people in two groups.
They developed in two languages, one statically typed, Java, and the other dynamically typed, Groovy.
Results revealed a positive impact of the use of types when these were used to document design decisions or when a high number of classes had to be identified by programmers.
On the other hand, for easier tasks, programmers developed faster in the dynamically typed language.
Our analysis suggest that programmers are aware of these tradeoffs and consider them when choosing whether to type or not their declarations.
In scenarios with small complexity, where neither readability nor stability is a concern, we show a lower usage of types.
Conversely, programmers type the interface of their modules very often, probably as a means to document their code.
This implicit documentation potentially improves the development time in more complex scenarios, such as those where programmers benefited from types in \cite{Hanenberg13}.

% In \cite{experiment_with_purity}, the author compares the performance of two groups of students asked to develop two small systems. 
% Both groups used a language developed by the author, Purity. 
% The only difference in the language used by the two groups was the type system.
% One group used a statically typed version of Purity while the other used a dynamically typed version of the same language.
% Results showed that the group using the dynamic version was significantly more productive than the other. 
% Similarly to this work, the author was able to compare two typing strategies directly while isolating any external factors. 
% However, it can be argued that these results may not represent well real life situations in the software industry. 
% This was a short duration study where students were used as examples of developers with no interaction with other programmers. 
% In our study, we try to get more relevant results when analyzing source code developed by programmers during their normal activities.
% 
% In a follow up study \cite{hanenberg_icpc}, the authors obtained opposite conclusions. 
% They compared the performance of two groups of developers in maintenance tasks. 
% The first group used Java, a statically typed language, and the other used Groovy, but restricting developers to use only untyped % declarations in order to simulate a dynamically typed version of Java.
% In this case, the group using the statically typed language, Java, was much more productive.
% This contradiction with a previous work reinforces the argument that the results of controlled studies are not reliable enough for this type % of study.

In experiments conducted in \cite{ruby_vs_druby}, the authors compare the performance of two groups working on small development tasks.
One group used Ruby, a dynamically typed language, while the other used DRuby, a statically typed version of Ruby. 
Results showed that the DRuby compiler rarely managed to capture any errors that were not already evident for programmers.
Most subjects involved in the study had previous experience with Ruby, which suggests that programmers get used to the lack of typing in their declarations.
Our analysis of the programmers' backgrounds supports this argument.
It shows that those programmers which have worked with dynamically typed languages in fact use types less often.








%
% CONCLUSION AND FUTURE WORK
%
\section{Conclusions and Future Work\label{conclusion}}
In this paper we conducted a large scale case study in order to investigate what are the factors that influence programmers of Groovy, an optionally typed language, on their choice whether to use types or not.
Our goal is to find what are the points of views of programmers about what typing strategies are more suitable to which contexts.
The main findings of this study are:
\begin{itemize}
	\item Groovy programmers type declarations that define the interface of modules more often than other declarations
	\item Types are less popular in test classes and script files
	\item Those programmers who have developed in at least one dynamically typed language use types less frequently than those who have only worked with statically typed languages
	\item Apparently, there is no influence of the size, age or level of activity of a project on how programmers use types
	\item In most projects, the files that change more frequently are also those files with a lower usage of types
\end{itemize}

We believe that these results are valuable to developers of programming languages and development tools who can base their designs on real user data.
Also, programmers can understand the tradeoffs between using or not types in their projects.
Finally, our results provide a different point of view and complement previous studies which analyzed typing strategies through the use of controlled experiments. 

We have raised many hypothesis trying to explain the causes for our results.
In future work we want to conduct controlled experiments and qualitative studies to evaluate such hypothesis.
In particular, we have started a new study with the goal of finding what are the impacts of the use of the documentation provided by types and the documentation provided by unit tests on the development time of maintenance tasks.
Another work in progress is a framework for the analysis of large numbers of source code repositories called ERA (Elastic Repository Analysis).
This framework is based on the artifacts we built in order to obtain and process our large dataset.
ERA will help other researchers to retrieve large numbers of projects from GItHub as well as process these projects quickly on the cloud using Amazon Web Services.

\acks
Many thank to Stefan Hanenberg for the invaluable contributions.

%
% BIBLIOGRAPHY
%
\bibliographystyle{abbrvnat}
\renewcommand{\bibfont}{\normalsize}
\begin{thebibliography}{}
\softraggedright

\bibitem[Tiobe Website(2013)]{tiobe}
Tiobe programming community index. http://www.tiobe.com/index.php/ content/paperinfo/tpci/index.html. Accessed in 23/09/2013.

\bibitem[Groovy(2013)]{groovy}
Groovy Programming Language. http://groovy.codehaus.org/. Accessed in 10/10/2013.

\bibitem[Bruce, K. (2002)]{bruce2002foundations}
Bruce, K. (2002). Foundations of object-oriented languages: types and semantics. MIT press.

\bibitem[Bruch, M., Monperrus, M., and Mezini, M. (2009)]{bruch2009learning}
Bruch, M., Monperrus, M., and Mezini, M. (2009). Learning from examples to improve code completion systems. In Proceedings of the the 7th joint meeting of the European software engineering conference and the ACM SIGSOFT symposium on The foundati- ons of software engineering, pages 213–222. ACM.

\bibitem[Cardelli, L. (1996]{type_systems}
Cardelli, L. (1996). Type systems. ACM Comput. Surv., 28(1):263–264.

\bibitem[Chang, M., Mathiske, B., Smith, E., Chaudhuri, A., Gal, A., Bebenita, M., Wimmer, C.]{jit}
Chang, M., Mathiske, B., Smith, E., Chaudhuri, A., Gal, A., Bebenita, M., Wimmer, C., and Franz, M. (2011). The impact of optional type information on jit compilation of dynamically typed languages. SIGPLAN Not., 47(2):13–24.

\bibitem[Daly, M. T., Sazawal, V., and Foster, J. S. (2009)]{ruby_vs_druby}
Daly, M. T., Sazawal, V., and Foster, J. S. (2009). Work In Progress: an Empirical Study of Static Typing in Ruby. In Workshop on Evaluation and Usability of Programming Languages and Tools (PLATEAU), Orlando, Florida.

\bibitem[Hanenberg, S. (2010)]{experiment_with_purity}
Hanenberg, S. (2010). An experiment about static and dynamic type systems: doubts about the positive impact of static type systems on development time. SIGPLAN Not., 45(10):22–35.

\bibitem[Kleinschmager, S., Hanenberg, S., Robbes, R., and Stefik, A. (2012)]{hanenberg_icpc}
Kleinschmager, S., Hanenberg, S., Robbes, R., and Stefik, A. (2012). Do static type systems improve the maintainability of software systems? An empirical study. 2012 20th IEEE International Conference on Program Comprehension (ICPC), pages 153– 162.

\bibitem[Lamport, L. and Paulson, L. C. (1999)]{should_your_specification_language_be_typed}
Lamport, L. and Paulson, L. C. (1999). Should your specification language be typed. ACM Trans. Program. Lang. Syst., 21(3):502–526.

\bibitem[Mayer,C.,Hanenberg,S.,Robbes,R.,Tanter,E ́.,andStefik,A.(2012)]{mayer2012static}
Mayer,C.,Hanenberg,S.,Robbes,R.,Tanter,E ́.,andStefik,A.(2012). Static type systems (sometimes) have a positive impact on the usability of undocumented software: An empirical evaluation. self, 18:5.

\bibitem[Pierce, B. (2002)]{types_and_programming_languages}
Pierce, B. (2002). Types and programming languages. MIT press.

\bibitem[Tratt, L. (2009)]{dynamically_typed_languages}
Tratt, L. (2009). Chapter 5 dynamically typed languages. volume 77 of Advances in Computers, pages 149 – 184. Elsevier.

\bibitem[Siek, Jeremy, and Walid Taha(2007)]{gradual_typing}
Siek, Jeremy, and Walid Taha. "Gradual typing for objects." ECOOP 2007–Object-Oriented Programming. Springer Berlin Heidelberg, 2007. 2-27.

\bibitem[Gray(2008)]{gray08}
Gray, K. E. Safe cross-language inheritance. In European Conference on Object-Oriented Programming (2008), pp. 52–75.

\bibitem[Gray(2011)]{gray11}
Gray, K. E. Interoperability in a scripted world: Putting inheritance \& prototypes together. In Foundations of Object-Oriented Languages (2011).

\bibitem[Gray(2005)]{gray05}
Gray, K. E., Findler, R. B.,Andflatt, M. Fine-grained interoperability through contracts and mirrors. InObject-Oriented Programming, Systems, Languages, and Applications(2005), pp. 231–245.

\bibitem[Siek(2007)]{siek07}
Siek, J.,Andtaha, W. Gradual typing for objects. In European Conference on Object-Oriented Programming(2007),pp. 2–27.

\bibitem[Takikawa(2012)]{takikawa12}
Takikawa, Asumu, et al. "Gradual typing for first-class classes." ACM SIGPLAN Notices. Vol. 47. No. 10. ACM, 2012.
\bibitem[Fowler(2010)]{fowler10}
Fowler, Martin. Domain-specific languages. Pearson Education, 2010.

\bibitem[Labovitz(1968)]{labovitz68}
Labovitz, Sanford. "Criteria for selecting a significance level: A note on the sacredness of. 05." The American Sociologist 3.3 (1968): 220-222.

% Types help on the definition of a contract
\bibitem[Meijer(2004)]{Meijer04}
Meijer, Erik, and Peter Drayton. "Static typing where possible, dynamic typing when needed: The end of the cold war between programming languages." OOPSLA, 2004.

\bibitem[Wadler(2004)]{Wadler04}
Wadler, Philip, and Robert Bruce Findler. "Well-typed programs can’t be blamed." Programming Languages and Systems. Springer Berlin Heidelberg, 2009. 1-16.

% Unfortunately, dynamically typed programming languages usually do not support the concept of DEC. Therefore we integrated DEC into the programming language Python by using a metaprogramming approach
\bibitem[Plosch(1997)]{Plosch97}
Plosch, Reinhold. "Design by contract for Python." Software Engineering Conference, 1997. Asia Pacific and International Computer Science Conference 1997. APSEC'97 and ICSC'97. Proceedings. IEEE, 1997.

%Traditional static type systems are very effective for verifying basic interface specifications
\bibitem[Flanagan(2006)]{Flanagan2006}
Flanagan, Cormac. "Hybrid type checking." ACM Sigplan Notices. Vol. 41. No. 1. ACM, 2006.

\bibitem[Meyer(1988)]{Meyer88}
Meyer, Bertrand. Object-oriented software construction. Vol. 2. New York: Prentice hall, 1988.

\bibitem[Furr(2009)]{Furr09}
Furr, Michael, et al. "Static type inference for Ruby." Proceedings of the 2009 ACM symposium on Applied Computing. ACM, 2009.

% Coupling is high for inheritance
% interaction coupling, component coupling and inheritance coupling
\bibitem[Chidamber(1994)]{Chidamber94}
Chidamber, Shyam R., and Chris F. Kemerer. "A metrics suite for object oriented design." Software Engineering, IEEE Transactions on 20.6 (1994): 476-493.

\bibitem[ISO25000(2004)]{Iso2004}
ISO, ISO, and IEC FCD. "25000, Software Engineering-Software Product Quality Requirements and Evaluation (SQuaRE)-Guide to SQuaRE." Geneva, International Organization for Standardization (2004).

\bibitem[Fenton(1998)]{Fenton1998}
Fenton, Norman E., and Shari Lawrence Pfleeger. Software metrics: a rigorous and practical approach. PWS Publishing Co., 1998.

\bibitem[Wohlin(2012)]{Wohlin2012}
Wohlin, Claes, et al. Experimentation in software engineering. Springer Publishing Company, Incorporated, 2012.

% Use of types in interfaces
\bibitem[Gannon(1977)]{Gannon77}
Gannon, J. D. An experimental evaluation of data type conventions. Commun. ACM 20, 8 (1977), 584–595.

\bibitem[Prechelt(1998)]{Prechelt98]}
Prechelt, L., and Tichy, W. F. A controlled experiment to assess the benefits of procedure argument type checking. IEEE Trans. Softw. Eng. 24, 4 (1998), 302–312

\bibitem[Hanenberg(2013)]{Hanenberg13}
Hanenberg, Stefan, et al. "An empirical study on the impact of static typing on software maintainability." Empirical Software Engineering (2013): 1-48.

\bibitem[Bruce(2002)]{Bruce2002}
Bruce, Kim B. Foundations of object-oriented languages: types and semantics. The MIT Press, 2002.

\bibitem[Curtis(1987)]{Curtis1987}
Curtis, BILL CURTIS. Five paradigms in the psychology of programming. MMC, 1987.

\bibitem[Beck(2003)]{Beck03}
Beck, Kent. Test-driven development: by example. Addison-Wesley Professional, 2003.

\bibitem[Meyerovich(2013)]{Meyerovich13}
Meyerovich, Leo A., and Ariel Rabkin. "Empirical Analysis of Programming Language Adoption." (2013).

\end{thebibliography}


\end{document}

%                       Revision History
%                       -------- -------
%  Date         Person  Ver.    Change
%  ----         ------  ----    ------

%  2013.06.29   TU      0.1--4  comments on permission/copyright notices

